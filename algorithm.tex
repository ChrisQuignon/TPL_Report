\subsection{Alorithm parallelization}
On the lowest level within the code one can manipulate the basic algorithms used in the software. In the world of video encoding in general there are many algorithms like the wavelet transformation or the macro block prediction. If there is excessive use of these algorithms, it might be considerable to evaluate these in parallel. Most of the common algorithms already have concurrent versions one can implement. For the generalizing approach of this essay i will spare the description of example algorithms. On a theoretical level they are mostly no different from complex software. They have I/O points and may be centered around data or tasks. They can be split in subtasks or get their data divided. The pitfalls of serial interdependencies and the difficulties in merging the data are the same as in complex software.\\
To decide if the algorithm parallelization will yield efficiency is strongly dependent of the software setup. In some cases, the single algorithms will show up on top of time intensive functions and most likely, they will be the most often called functions. When the core algorithms dominate the rest of the program, the might get split to some of the SPEs. Most of these algorithms work in a functional way, consuming the input and generating one new output with compact code. Thus they are not ideal for multiple SPE parallelization, but utilize one SPE ideal. As before, on other architectures like the MISD (Multiple Instruction, Single Data) the decision on what to parallelize will differ.
%\subsection{subsection}
%\subsubsection{subsubsection}
%\paragraph{paragraph}
%\subparagraph{subparagraph}